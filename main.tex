\documentclass[addpoints]{exam}

\usepackage{hyperref}

% Header and footer.
\pagestyle{headandfoot}
\runningheadrule
\runningfootrule
\runningheader{CS 201 DS II}{Homework 2}{Spring 2018}
\runningfooter{}{Page \thepage\ of \numpages}{}
\firstpageheader{}{}{}

\qformat{{\large\bf Exercise \thequestiontitle}\hfill[\totalpoints\ points]}
\boxedpoints
\printanswers

\title{Habib University\\CS 201 Data Structures II\\Spring 2018}
\author{Don't Grade Me}  % replace with your ID, e.g. sh01703
\date{Homework 2\\Due: 19h; Monday, 12 Feb}

\begin{document}
\maketitle

\begin{questions}

  \titledquestion{4.8*}[10]
  The {\tt find(x)} method in a SkiplistSet sometimes performs redundant comparisons; these occur when $x$ is compared to the same value more than once. They can occur when, for some node, {\tt u}, {\tt u.next[r] = u.next[r−1]}. Show how these redundant comparisons happen and explain how {\tt find(x)} can be modified so that they are avoided.
\begin{solution}
  % Write your solution here
\end{solution}

\titledquestion{4.9*}[5]
Explain how we can implement a version of a skiplist that implements the SSet interface, but also allows fast access to elements by rank. That is, it also supports the function $get(i)$, which returns the element whose rank is $i$ in $O(\log n)$ expected time. (The rank of an element $x$ in an SSet is the number of elements in the SSet that are less than $x$.)
\begin{solution}
  % Write your solution here
\end{solution}

\titledquestion{4.14* (challenge)}[0]
Using an {\tt SSet} as your underlying structure, design an application that reads a (large) text file and allows you to search, interactively, for any substring contained in the text. As the user types their query, a matching part of the text (if any) should appear as a result.

\noindent\underline{Hint 1}: Every substring is a prefix of some suffix, so it suffices to store all suffixes of the text file.

\noindent\underline{Hint 2}: Any suffix can be represented compactly as a single integer indi- cating where the suffix begins in the text.

Test your application on some large texts, such as some of the books available at \href{http://www.gutenberg.org/}{Project Gutenberg}. If done correctly, your application will be very responsive; there should be no noticeable lag between typing keystrokes and seeing the results.
\begin{solution}
  % Write your solution here
\end{solution}

\titledquestion{5.1}[5]
As in the book.
\begin{solution}
  % Write your solution here
\end{solution}

\titledquestion{5.2}
\begin{parts}
  \part[5] As in the book.
  \begin{solution}
    % Write your solution here
  \end{solution}

    \part[5] As in the book.
  \begin{solution}
    % Write your solution here
  \end{solution}
\end{parts}

\titledquestion{5.5}[10]
As in the book.
\begin{solution}
  % Write your solution here
\end{solution}

\titledquestion{5.9}[10]
As in the book.
\begin{solution}
  % Write your solution here
\end{solution}

\titledquestion{5.10}[10]
As in the book.
\begin{solution}
  % Write your solution here
\end{solution}
\end{questions}

* - The question has been modified from the one in the book to exclude implementation. ``Design'' in a question here means to write pseudocode.

\end{document}